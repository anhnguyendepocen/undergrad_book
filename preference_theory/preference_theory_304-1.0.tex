%% This document created by Scientific Word (R) Version 3.0

\documentclass[12pt]{article}
\usepackage{graphicx}
\usepackage{amsmath}
\usepackage{amsfonts}
\usepackage{amssymb}
%TCIDATA{OutputFilter=latex2.dll}
%TCIDATA{CSTFile=LaTeX article (bright).cst}
%TCIDATA{Created=Thu Aug 26 15:33:05 2004}
%TCIDATA{LastRevised=Fri Aug 27 17:17:01 2004}
%TCIDATA{<META NAME="GraphicsSave" CONTENT="32">}
%TCIDATA{<META NAME="DocumentShell" CONTENT="General\Blank Document">}
%TCIDATA{Language=American English}
\newtheorem{theorem}{Theorem}
\newtheorem{acknowledgement}[theorem]{Acknowledgement}
\newtheorem{algorithm}[theorem]{Algorithm}
\newtheorem{assumption}[theorem]{Assumption}
\newtheorem{axiom}[theorem]{Axiom}
\newtheorem{case}[theorem]{Case}
\newtheorem{claim}[theorem]{Claim}
\newtheorem{conclusion}[theorem]{Conclusion}
\newtheorem{condition}[theorem]{Condition}
\newtheorem{conjecture}[theorem]{Conjecture}
\newtheorem{corollary}[theorem]{Corollary}
\newtheorem{criterion}[theorem]{Criterion}
\newtheorem{definition}[theorem]{Definition}
\newtheorem{example}[theorem]{Example}
\newtheorem{exercise}[theorem]{Exercise}
\newtheorem{lemma}[theorem]{Lemma}
\newtheorem{notation}[theorem]{Notation}
\newtheorem{problem}[theorem]{Problem}
\newtheorem{proposition}[theorem]{Proposition}
\newtheorem{remark}[theorem]{Remark}
\newtheorem{solution}[theorem]{Solution}
\newtheorem{summary}[theorem]{Summary}
\newenvironment{proof}[1][Proof]{\textbf{#1.} }{\ \rule{0.5em}{0.5em}}

\begin{document}
\section{Preferences}

The foundation of all choice theory in economics is something called a
\emph{preference relation}. The idea is that if I present you with a pair of
alternatives, then you could tell me which one you prefer, or possibly that
you are indifferent between them. The word 'prefer' has different meanings in
different contexts. For example, if I ask you whether you would prefer to see
a movie or go to a hockey game your preference is expressing something about
which you would enjoy. If I ask you whether you would like to have the
Olympics in your local city, your preference may express something about what
you think is best for everyone, or possibly something about what you think you
are supposed to say. Sometimes you really can't say that one alternative is
better than another. For example you might be equally happy with a ham
sandwich or a tuna sandwich. If I allow for that possibility, it is hard to
imagine a situation where you wouldn't be able to say something.

If I am trying to think about your choice behavior and how I might understand
it, I could begin by trying to imagine all the alternatives that you could
possibly choose. I would collect them together in a big set $X$. Then I could
go about choosing different pairs of alternatives in $X$ and asking you to
express your opinion about which of the two you prefer. Eventually (as long as
you didn't get tired of answering questions) I could learn which alternative
you preferred among any pair of alternatives in $X$. This collection of
information is your preference relation over $X$.

The set $X$ could be very general. For example, you might have guessed that we
are going to be talking about preference relations over collections of
possible consumption bundles. There is no need to stop there. Much of modern
microeconomic theory arises from thinking about preferences over things like
political parties, environmental policies, business strategies, location
decisions, and so on.

There are many kinds of preference relations you will encounter if you
continue studying economics, but the most widely applied reasoning in
economics assumes that preference relations have two properties - first, they
must be \emph{complete} in the sense that for \emph{any} pair of alternatives
in $X$, either you prefer one or the other, or are indifferent. There are some
interesting preference relations that are incomplete, but let's leave that for
the moment and concentrate on another problem. Your preference relation could
be 'odd'. For example, suppose you like the Liberals more than the
Conservatives because they are more socially progressive. You might like the
NDP more than the Liberals for the same reason. However, you may prefer the
Conservatives to the NDP because they are more fiscally responsible. Ignoring
any other parties, then you have just expressed a complete and reasonable
preference relation over the political parties. It does present something of a
problem when you are trying to vote. You can't vote Conservative because you
prefer the Liberals to the Conservatives, you can't vote Liberal because you
prefer the NDP to the Liberals. Unfortunately you can't vote NDP either
because you prefer the Conservatives to the NDP.\footnote{I suppose this could
explain why so many people don't vote.}

We have a word for this kind of preference relation in economics, it is called
an \emph{intransitive} preference relation. To put this another way, a
\emph{transitive }preference relation is one such that for any 3 alternatives
$x$, $y$, and $z$ in $X$, if $x$ is preferred to $y$ and $y$ is preferred to
$z$, then it must be the case that $x$ is preferred to $z$. A complete
transitive preference relation is called a \emph{rational preference relation}.

In fact, I have just described to you what rationality means in economics.
A\ person is said to be rational in a particular economic environment if they
have a complete and transitive preference relation over the alternatives that
they face in that environment. In particular, it doesn't mean that people are
greedy or self interested. It doesn't mean that they are super sophisticated
calculators. It just means that they can express opinions about pairs of alternatives.

\subsection{Behavior}

So how do economists go about predicting what people will do? All they say is
that whatever alternative $x$ is actually chosen from $X$, then there cannot be
another alternative in $X$ that is preferred to $x$. It is true that in
experiments, people sometimes exhibit intransitive preferences (though they
quickly change their behavior when this is pointed out to them). There are
also situations in which it seems impossible for people to make a choice. For
the most part though, assuming that people are rational (have a complete
transitive preference relation) is pretty innocuous.

It might also occur to you that if you accept that people are rational
decision makers, then you can't really get yourself in too much trouble.
I\ never said what these preference relations had to look like. To assert that
an individual chooses the alternative that he or she most prefers is almost
tautological. The real content of economic theory involves restrictions it
imposes on $X$ and on the preference relation over $X$. Its failures and
successes having nothing to do with the assumption of rationality.

I want to show you the theorem that has transformed economics into a social
science that is unlike most others. Before this, let me digress on another
important idea. Introductory economics courses focus on consumption and
consumption bundles. A consumption bundle is a pair $\left(  x,y\right)  $
where the first component of this vector is some quantity that you consume of
one good (just call it good $x$ for short), and the second component is the
quantity you consume of the second good. Consumption doesn't generate
happiness or utility or utils or anything like that. If we follow your first
year course, and imagine that good $x$ has a price $p$ and good $y$ has a
price $q$, and that you have $W$ to spend, then the consumer faces a
set of alternatives $X$ which consists of all pairs $\left(  x,y\right)  $
whose cost is less than or equal to $W$, i.e.,%
\[
X\equiv\left\{  \left(  x,y\right)  \in\mathbb{R}_{+}^{2}:px+qy\leq
W\right\}
\]
Here $\mathbb{R}_{+}^{2}$ is the set of all vectors with two non-negative
components. Read the colon to mean ''such that''.

Well, since we have a set of alternatives, it is pretty safe to assume that
for any pair of alternatives (a pair of alternatives is a pair of vectors
$\left(  x,y\right)  $ and $\left(  x^{\prime},y^{\prime}\right)  $ here), the
consumer can express a preference between them. Suppose for the moment that we
could get the consumer to tell us what his or her preference relation is. But
now we face a small problem. Suppose the consumer tells us that he prefers
$\left(  x,y\right)  $ to $\left(  x^{\prime},y^{\prime}\right)  $. Suppose
that we now look at another budget set $X^{\prime}$ where prices are
$p^{\prime}$ and $q^{\prime}$, and maybe income is $W^{\prime}$. Let's pick
this new set so that it contains both $\left(  x,y\right)  $ and $\left(
x^{\prime},y^{\prime}\right)  $. Do we really need to ask the consumer if he
prefers $\left(  x,y\right)  $ to $\left(  x^{\prime},y^{\prime}\right)  $ in
this new set? Of course his preference could well change. People have no use
for telephones unless other people have telephones. The change in income might
mean that others can buy phones. The price changes might signal changes in
quality of the goods that he is buying (suppose $x$ and $y$ are stocks or
bonds or something like that).

Now we begin to impose some restrictions of preferences and economic theory
begins to have some content (of course, we also study what happens when
preference relations change with prices and income). We are going to assume
that if $\left(  x,y\right)  $ and $\left(  x^{\prime},y^{\prime}\right)  $
are in both $X$ and $X^{\prime}$ and if $\left(  x,y\right)  $ is preferred to
$\left(  x^{\prime},y^{\prime}\right)  $ in the preference relation relative
to $X$, then it must also be preferred in the preference relation relative to
$X^{\prime}$.

The important point is that the assumption that our consumer was rational
imposed no restriction whatsoever on his behavior. This added assumption does
restrict what we should expect to see him do. For example, suppose that we
could run a long series of experiments in which our consumer is repeatedly
asked to choose something from $X$ and that he consistently chooses $\left(
x,y\right)  $. If our assumption is true, then it would be highly unlikely
that if we had him choose repeatedly from $X^{\prime}$ that he would
consistently pick $\left(  x^{\prime},y^{\prime}\right)  $.\footnote{He might
do this once if he were indifferent, but would probably not do it consistently
if he were indifferent.} The predictive content of the theory comes from the
assumption that his preference relation is independent of the prices and
income that he faces, not from the assumption that he is rational.

You will see this repeatedly in economics - we will impose restrictions on $X$
and the preference relation over it, then make predictions (and test them). If
you want to argue about economics the idea is to understand these restrictions
and criticize them. It is a waste of time to argue about whether or not
consumers are rational.

\subsection{Indifference Curves}

So let's continue with first year economics. Since preference relations (let's 
just say preferences from now on) are assumed to be independent of prices and
income, we could sensibly take the consumer's preference relation and collect
together \emph{all }the consumption bundles $\left(  x^{\prime},y^{\prime
}\right)  $ which are indifferent to some bundle $\left(  x,y\right)
$.\footnote{To be formal, we could say that $\left(  x,y\right)  $ is
indifferent to $\left(  x^{\prime},y^{\prime}\right)  $ if $\left(
x,y\right)  $ is at least as good as $\left(  x^{\prime},y^{\prime}\right)  $
and at the same time $\left(  x^{\prime},y^{\prime}\right)  $ is at least as
good as $\left(  x,y\right)  $.} As you remember from your first year course,
this collection of consumption bundles is called an \emph{indifference curve}.
Please note that the indifference curve comes directly from the preference
relation and has nothing to do with utils or satisfaction of anything like
that. Since we can construct an indifference curve for any consumption bundle,
there is really a \emph{family} of indifference curves. 

Pick two indifference
curves in this family, say $C_{1}$ and $C_{2}$ and choose a bundle $\left(
x,y\right)  $ from $C_{1}$ (which is itself a set) and $\left(  x^{\prime
},y^{\prime}\right)  $ from $C_{2}$. If $\left(  x,y\right)  $ is preferred to $\left(  x^{\prime},y^{\prime
}\right)  $ then we say that the indifference curve $C_{1}$ is \emph{higher
than }$C_{2}$. Then of course, any bundle in $C_{1}$ will be preferred to any
bundle in $C_{2}$. There isn't much that can be said about indifference curves
at this point except that when a consumer is rational, two distinct
indifference curves can't have any point in common. To see this suppose that
$C_{1}$ is higher than $C_{2}$. Let $\left(  x^{\prime\prime},y^{\prime\prime
}\right)  $ be the point that the curves have in common, with $\left(
x,y\right)  $ in $C_{1}$ and $\left(  x^{\prime},y^{\prime}\right)  $ in
$C_{2}$. Then $\left(  x^{\prime},y^{\prime}\right)  $ is at least as good as
$\left(  x^{\prime\prime},y^{\prime\prime}\right)  $ since both are in $C_{2}%
$. $\left(  x^{\prime\prime},y^{\prime\prime}\right)  $ is at least as good as
$\left(  x,y\right)  $ since both are in $C_{1}$. Now transitivity requires
that $\left(  x^{\prime},y^{\prime}\right)  $ be at least as good as $\left(
x,y\right)  $ which is false if the consumer is rational.\footnote{A small
digression - this simple argument is an example of a line of reasoning that
you will see often in economics. If you want to show that some property A
implies that another property B must be true, try to show that if $B$ isn't
true, then A can't be true either. This is called a proof \emph{by
contradiction.} Here we wanted to show that if a preference relation is
transitive (A) then a pair of indifference curves couldn't cross (B). We
showed that if the curves did cross, the preference relation couldn't be transitive.}

At this point, we could try to describe graphic properties of the indifference
curves. If we started to do that, we would end up spending considerable time
trying to absorb graphic formalism and end up saying what we could have said
with words. So it is time for me to introduce the theorem that makes economics work.

Write the preference relation as $\succeq$, meaning that $\left(  x,y\right)
\succeq\left(  x^{\prime},y^{\prime}\right)  $ whenever $\left(  x,y\right)  $
is preferred to $\left(  x^{\prime},y^{\prime}\right)  $. A \emph{utility
function} is a relation that converts each bundle $\left(  x,y\right)  $ into
a corresponding utility value or number. The utility function $u$ represents
the preference relation $\succeq$ as long as $u\left(  x,y\right)  \geq
u\left(  x^{\prime},y^{\prime}\right)  $ if and only if $\left(  x,y\right)
\succeq\left(  x^{\prime},y^{\prime}\right)  $. If we happened to be able to
find a utility function to represent a preference relation then we would have
a big leg up. To predict what a consumer will do so far, we need to scan all
pairs of consumption bundles until we find a bundle such that no other bundle
is preferred to it. This makes for a lot of tedious pairwise comparisons.
There isn't any obvious reason why this sort of reasoning is going to help us
understand behavior. If preferences are represented by a utility function, we
could take the function and find the bundle that produced the highest utility
number in the set of alternatives. That would be relatively easy because we
could use all the standard mathematical tricks we know about maximizing
functions (like setting derivatives to zero and so on).

Yet the utility function yields something far more important. As I mentioned
above, the content of economic theory doesn't come from the rationality
assumption. It comes from imposing restrictions on the preference relation and
the feasible set. It is difficult to formulate ideas about preference
relations since they are relative complex objects. On the other hand, it is
much easier to impose and understand restrictions on utility functions.

Assuming that people have utility functions which they maximize is just about
the last thing we want to do. If we did that, then all the people who accuse
economists of being irrelevant because they assume that consumers are
'rational' would have a good point. We would be guilty of predicting behavior
by assuming that people do something that they obviously don't.

So why use a utility function? We need to add one important restriction on
preference relations, and one simplifying restriction.\footnote{Simplifying
means that I could make the same argument I am about to make without the
restriction, but it would take me a lot longer.} The simplifying restriction
is that our consumer likes more of both goods - i.e., if $\left(  x,y\right)
$ and $\left(  x^{\prime},y^{\prime}\right)  $ are such that $x\geq x^{\prime
}$ and $y\geq y^{\prime}$ and at least one of these inequalities is strict,
then $\left(  x,y\right)  \succeq\left(  x^{\prime},y^{\prime}\right)  $ but
not the other way around. Having more of any good makes the consumer strictly
better off. For short, let's say that such a preference relation is
\emph{monotonic}.

Now for the important restriction. Let $B=\left\{  \left(  x^{\prime
},y^{\prime}\right)  \in\mathbb{R}_{+}^{2}:\left(  x^{\prime},y^{\prime
}\right)  \succeq\left(  x,y\right)  \right\}  $ and $W=\left\{  \left(
x^{\prime},y^{\prime}\right)  \in\mathbb{R}_{+}^{2}:\left(  x,y\right)
\succeq\left(  x^{\prime},y^{\prime}\right)  \right\}  $. Then both $B$ and
$W$ are \emph{closed} sets\footnote{A closed set is one for which any
convergent sequence of points in the set converges to a point in the set.}. If
these sets are closed for any $\left(  x,y\right)  \in\mathbb{R}_{+}^{2}$ then
the preference relation is said to be \emph{continuous. }Now the following
important theorem is true:

\begin{theorem}
Let $\succeq$ be a continuous and monotonic rational preference relation. Then
there exists a utility function $u$ which represents the preference relation
$\succeq$.
\end{theorem}

\begin{proof}
We are going to prove this constructively by actually making up the function.
First some preliminaries. Let $Z$ represent the 45$^{0}$ line (i.e., the set
of all points in $\mathbb{R}_{+}^{2}$ which have the same horizontal and
vertical coordinate). Let $\left(  x,y\right)  $ be any consumption bundle.
Let $\varepsilon>0$ be a small positive number. The bundle $\left(
\max\left[  x,y\right]  +\varepsilon,\max\left[  x,y\right]  +\varepsilon
\right)  $ is in $Z$ and is strictly preferred to $\left(  x,y\right)  $ by
the fact that preferences are monotonic. Similarly $\left(  x,y\right)  $ is
preferred to $\left(  \min\left[  x,y\right]  -\varepsilon,\min\left[
x,y\right]  -\varepsilon\right)  $ by monotonicity. So the sets $B$ and $W$
are both non-empty. As preferences are continuous, these sets are both closed.
This lets us deduce that the sets $P^{+}=B\cap Z$ and $P^{-}=W\cap Z$ are both
closed as the intersection of closed sets. In Figure \ref{g1} the set $P^{+}$
is marked in red. It is the intersection of the 45$^{0}$ line and the set $B$
consisting of all bundles that are preferred to $\left(  x,y\right)  $. The
set $P^{-}$ is marked in blue in the figure.

Now the sets $P^{+}$ and $P^{-}$ are made up of bundles (in $\mathbb{R}%
_{+}^{2}$) that have the same horizontal and vertical component. So we can
associated each bundle in $z$ with this common component, which is just a
positive real number. Since each bundle $z\in Z$ either has $z\succeq\left(
x,y\right)  $ or $\left(  x,y\right)  \succeq z$ by the completeness of
preferences, (recall that completeness is part of rationality) each point in
$z$ is either in $P^{+}$ or $P^{-}$. Each point in $P^{+}$ or $P^{-}$ is also
in $Z$ by construction, so $Z=P^{+}\cup P^{-}$.

By happy coincidence\footnote{If you believe it is a coincidence, I have a
bridge to sell you.} $P^{+}$ and $P^{-}$ share exactly one point in common.
Part of the argument for this is an arcane point in set theory. Since
$P^{+}\cup P^{-}$ is all of $Z$, if they don't share a common point, then
$P^{-}$ must be the complement of $P^{+}$ in $Z$. Since the complement of a
closed set is open, $P^{-}$ would have to be open which it cannot be. So there
must be at least one common point. Could there be two? Again, suppose there
were, say $z$ and $z^{\prime}$. They are both in $Z$ so they are both on the
45$^{0}$ line. If they are distinct then, say, $z>>z^{\prime}$ (meaning each
component of $z$ is strictly larger than the corresponding component of
$z^{\prime}$). Then by monotonicity $z\succeq z^{\prime}$ but not the other
way around. Then by transitivity $\left(  x,y\right)  \succeq z^{\prime}$ but
not the other way around. But this can't be since $z^{\prime}\in P^{+}$.

All this work leads to the conclusion that for every bundle $\left(
x,y\right)  $ we can find a point on the 45$^{0}$ line which is indifferent to
it. Let's call the common coordinate of this point the \emph{utility }$u\left(
x,y\right)  $ associated with the bundle $\left(  x,y\right)  $ (this probably
emphasizes the point that utility is measured as some number of goods, not as
utils or satisfaction).

Finally, all we need to do is check that this utility function $u\left(
x,y\right)  $ actually represents preferences. This is pretty straightforward.
For example, if $u\left(  x,y\right)  \geq u\left(  x^{\prime},y^{\prime
}\right)  $ then the $z$ associated with $\left(  x,y\right)  $ has a bigger
common component than the $z^{\prime}$ associated with $\left(  x^{\prime
},y^{\prime}\right)  $. Then $\left(  x,y\right)  \succeq z$ (since $z\in
P^{-}$ for $\left(  x,y\right)  $) $\succeq z^{\prime}$ (by monotonicity)
$\succeq\left(  x^{\prime},y^{\prime}\right)  $ (since $z^{\prime}\in P^{+}$
for $\left(  x^{\prime},y^{\prime}\right)  $). The other direction is just as easy.
\end{proof}%

%TCIMACRO{\FRAME{ftbpFU}{2.1032in}{1.6812in}{0pt}{\Qcb{The sets $P^{+}$ and
%$P^{-}$}}{\Qlb{g1}}{preferences_fig1.mps}%
%{\special{ language "Scientific Word";  type "GRAPHIC";
%maintain-aspect-ratio TRUE;  display "USEDEF";  valid_file "F";
%width 2.1032in;  height 1.6812in;  depth 0pt;  original-width 2.0029in;
%original-height 1.5956in;  cropleft "0";  croptop "1";  cropright "1";
%cropbottom "0";  filename 'preferences_fig1.mps';file-properties "XNPEU";}}}%
%BeginExpansion
\begin{figure}
[ptb]
\begin{center}
\includegraphics[
natheight=1.595600in,
natwidth=2.002900in,
height=1.6812in,
width=2.1032in
]%
{preferences_fig1.mps}%
\caption{The sets $P^{+}$ and $P^{-}$}%
\label{g1}%
\end{center}
\end{figure}
%EndExpansion

So let's collect our thoughts for a moment. When a consumer chooses a bundle
from some budget, set she picks something such that if we offer her some other
bundle from the same budget set, she will not want it. If her preferences are
transitive and complete (and continuous) it will \emph{appear to be the case}
that she is choosing a bundle to maximize a utility function subject to the
budget constraint. In the consumer's own mind, there is no such thing as
utility: rational utility maximization is an implication of simpler properties
of consumer behavior. Nor is it assumed that there is any numerical way to
measure happiness or satisfaction. These simply aren't parts of modern
microeconomic theory.

\subsection{Economic Modelling}

Why was this theorem so important? Well it shows first that economic
methodology itself doesn't rely on grand assumptions about human behavior. Of
course, when we impose restrictions on the preference relation or the set of
feasible alternatives, we are making assumptions. These assumptions are part
of what we call economic \emph{models}. When we formulate an economic model,
we try to extract all the implications of the restrictions. These restrictions
are predictions the model makes. We can collect data about the choices
consumers actually do make, to check whether these predictions are right. When
they are wrong, we know we need to reformulate the model (or change some of
the restrictions).

The second thing is shows is that we can extract these restrictions using some
fairly basic mathematical tools, like the theory of optimization (and of
course, the dreaded calculus). The mathematization of economics occurred in
the late 50's and has had a remarkable impact on the way economists interact.
To use mathematics, it is necessary that the concepts, sets, and functions
involved be very precisely defined. There is no room for interpretation
(though certainly there is room to fine tune and modify concepts). An economic
concept must mean the same thing to everyone.

This has had an impact that you might not expect. Anyone who understands basic
mathematics should be able to understand the most advanced ideas in economic
theory. Oddly enough mathematics makes economics very inclusive.\footnote{You
might like to compare the definition of utility I\ have given above with
definitions you will hear for important concepts like capitalism or post
modernism.} This has had great benefits for economist, since other fields have
been moving in much the same direction. Computer science, biology, ecology,
environmental science, all use methods similar to those used by economists.
The level of interaction among practitioners in these different fields is
increasing to the enrichment of all.

Most of this course tries to develop the mathematical and conceptual tools you
need to formulate and analyze economic models on your own. As we go about
this, you will see some models that have worked out pretty well in the sense
that they give very good insight into some pretty applied problems. You will
also get a chance to see some models that don't work so well. These 'failures'
give a good deal of insight into how theoretical and empirical work interact.
Though these applications are important in the overall scheme of things, they
are not the main focus of the course. It is the art of building the models
themselves that is the concern here. Once you begin to appreciate this
approach, your subsequent studies in more applied areas will make more sense.
\end{document}