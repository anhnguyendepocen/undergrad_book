%% This document created by Scientific Word (R) Version 3.0

\documentclass{article}
\usepackage{graphicx}
\usepackage{amsmath}
\usepackage{amsfonts}
\usepackage{amssymb}
%TCIDATA{OutputFilter=latex2.dll}
%TCIDATA{CSTFile=LaTeX article (bright).cst}
%TCIDATA{Created=Mon Nov 24 12:24:47 2003}
%TCIDATA{LastRevised=Fri Nov 28 11:39:52 2003}
%TCIDATA{<META NAME="GraphicsSave" CONTENT="32">}
%TCIDATA{<META NAME="DocumentShell" CONTENT="General\Blank Document">}
%TCIDATA{Language=American English}
\newtheorem{theorem}{Theorem}
\newtheorem{acknowledgement}[theorem]{Acknowledgement}
\newtheorem{algorithm}[theorem]{Algorithm}
\newtheorem{axiom}[theorem]{Axiom}
\newtheorem{case}[theorem]{Case}
\newtheorem{claim}[theorem]{Claim}
\newtheorem{conclusion}[theorem]{Conclusion}
\newtheorem{condition}[theorem]{Condition}
\newtheorem{conjecture}[theorem]{Conjecture}
\newtheorem{corollary}[theorem]{Corollary}
\newtheorem{criterion}[theorem]{Criterion}
\newtheorem{definition}[theorem]{Definition}
\newtheorem{example}[theorem]{Example}
\newtheorem{exercise}[theorem]{Exercise}
\newtheorem{lemma}[theorem]{Lemma}
\newtheorem{notation}[theorem]{Notation}
\newtheorem{problem}[theorem]{Problem}
\newtheorem{proposition}[theorem]{Proposition}
\newtheorem{remark}[theorem]{Remark}
\newtheorem{solution}[theorem]{Solution}
\newtheorem{summary}[theorem]{Summary}
\newenvironment{proof}[1][Proof]{\textbf{#1.} }{\ \rule{0.5em}{0.5em}}

\begin{document}
\title{Public Goods}
\author{Michael Peters}
\date{\today}
\maketitle

\section{Introduction}
A \emph{public good} is a good having the property that the total output of
the good is enjoyed by everyone. In contrast, a \emph{private good} has the
property that if I consume it, you cannot. Some of the more important goods in
the modern economics are actually public. For example an idea or a bit of
information is a public good. Whether or not you know the idea, or have the
information does not impact on my ability to know the same thing or have the
same information (though whether or not you have the same information as I do
may determine whether or not I can make money off you). A music file shared on
my computer is a public good - if you take it from me my ability to enjoy it
is not affected at all. A theorem, like the first welfare theorem that we
studied last week -->CHAPTER NUMBER ?, is a public good - I do not forget it when you learn it.
Contrast this with my car or my lunch which are both private goods - if you
take them from me I cannot enjoy them at all.

This note CHAPTER ? describes an equilibrium for the \emph{voluntary contribution game},
which is a common way to think about how public goods are provided. It
explained why the amount of the public good that is produced in the voluntary
contribution game is too low (because the outcome is not pareto optimal -
there is another outcome that will make everyone better off). The note CHAPTER ? then
explains how public goods can be priced to ensure pareto optimality in supply.

\subsection{The voluntary contribution game}

Suppose there are two goods, one public ($y$) and one private ($x$). Let
$f\left(  x\right)  $ denote the amount of the public good that can be
produced from $x$ units of the private good. Suppose there are two consumers
with utility functions $u_{1}\left(  x,y\right)  $ and $u_{2}\left(
x,y\right)  $ respectively. Their endowments of the private good are
$\omega_{1}$ and $\omega_{2}$. The set of points $\left\{  \left(  x,y\right)
:y=f\left(  \omega_{1}+\omega_{2}-x\right)  \right\}  $ THIS IS INCONSISTENT WITH THE PREVIOUS CHAPTER, ``PROFIT MAXIMIZATION", WHICH USES $\left\{  \left(  x,y\right)
:y=f\left( x -\omega_{1}-\omega_{2}\right)  \right\}  $ is the
\emph{production possibilities frontier}.

If the first consumer decides to consume $x_{1}$ (and devote the rest of his
endowment $\omega_{1}$ to production of the public good) while consumer $2$
decides to consume $x_{2}$ the utilities of each of the consumers are given by%
\[
u_{1}\left(  x_{1},f\left(  \omega_{1}+\omega_{2}-x_{2}-x_{2}\right)
\right)
\]
for consumer $1$ and%
\[
u_{2}\left(  x_{2},f\left(  \omega_{1}+\omega_{2}-x_{1}-x_{2}\right)
\right)
\]
for consumer $2$. The important point is that if consumer $1$ say decides to
consume a bit less of the private good and produce a bit more of the public
good, then consumer $2$ will enjoy the additional public good too without any
cost at all.

One way to predict how much of the private good each consumer will choose is
to focus on the equilibrium of the voluntary contribution game. Each of the
consumers simply picks the amount of the private good they want on their own.
This is a bit hard to do because the amount that each consumer will choose to
contribute depends on how much they expect the other consumer to contribute. A
good way to do this is to use a \emph{Nash equilibrium} in place of a
Walrasian equilibrium. Instead of taking prices to be fixed, each consumer
takes the contribution of the other consumer to be fixed and chooses the
contribution that maximizes his utility given this expectation. In a Walrasian
equilibrium, when a consumer acts as if he believes that prices are fixed, he
has to be physically able to purchase the bundle that maximizes his utility at
these prices. That is why the price expectations have to be such that markets
clear. Analogously, when the consumer chooses his optimal contribution given a
fixed expectation about the contribution of the other consumer, he has to end
up with exactly the amount of the public good that he expected to get.

Formally, a Nash equilibrium for the voluntary contribution game is a pair of
private consumptions $x_{1}^{\ast}$ and $x_{2}^{\ast}$ such that%
\[
u_{1}\left(  x_{1}^{\ast},f\left(  \omega_{1}+\omega_{2}-x_{2}^{\ast}%
-x_{1}^{\ast}\right)  \right)  \geq u_{1}\left(  x^{\prime},f\left(
\omega_{1}+\omega_{2}-x_{2}^{\ast}-x^{\prime}\right)  \right)
\]
for any alternative contribution $x^{\prime}\in\left[  0,\omega_{1}\right]  $
and%
\[
u_{2}\left(  x_{2}^{\ast},f\left(  \omega_{1}+\omega_{2}-x_{2}^{\ast}%
-x_{1}^{\ast}\right)  \right)  \geq u_{2}\left(  x^{\prime},f\left(
\omega_{1}+\omega_{2}-x_{1}^{\ast}-x^{\prime}\right)  \right)
\]
for any alternative contribution $x^{\prime}\in\left[  0,\omega_{2}\right]  $.

One way to view the outcome of this game is given in figure $1$ where the
chosen consumptions of the two consumers are $x_{1}^{\ast}$ and $x_{2}^{\ast}%
$. The 'budget line' that consumer $1$ faces, for example, when consumer $2$
chooses consumption $x_{2}^{\ast}$ is the set of all pairs $\left\{  \left(
x,y\right)  :y=f\left(  \omega_{1}+\omega_{2}-x_{2}^{\ast}-x\right)  \right\}
$. The slope of this is exactly the same as the slope of the production
possibilities frontier at the point $\left(  x_{1}^{\ast},x_{2}^{\ast}%
,y^{\ast}\right)  $. The same is true for consumer $2$. So in the equilibrium
of the voluntary contribution game, each consumer has the same marginal rate
of substitution, and both of these are equal to the marginal rate of
transformation in production.  ALL THE FIGURES SHOULD BE LABELED AND APPEAR MORE OR LESS WHERE THEY ARE MENTIONED (I.E. FIGURE SHOULD APPEAR ON PG 2%
%TCIMACRO{\FRAME{ftbpF}{4.0119in}{2.4197in}{0pt}{}{}{public_goods_fig1.eps}%
%{\special{ language "Scientific Word";  type "GRAPHIC";
%maintain-aspect-ratio TRUE;  display "USEDEF";  valid_file "F";
%width 4.0119in;  height 2.4197in;  depth 0pt;  original-width 3.845in;
%original-height 2.3082in;  cropleft "0";  croptop "1";  cropright "1";
%cropbottom "0";  filename 'public_goods_fig1.eps';file-properties "XNPEU";}}}%
%BeginExpansion
\begin{figure}
[ptb]
\begin{center}
\includegraphics[
height=2.4197in,
width=4.0119in
]%
{public_goods_fig1.eps}%
\end{center}
\end{figure}
%EndExpansion
With private goods this is exactly what you want. Recall that in the Edgeworth
box THIS TERM SHOULD BE EXPLAINED IN THE PREVIOUS CHAPTER, ``PROFIT MAXIMIZATION" both consumers indifference curves were tangent and the common slope of
their indifference curves was equal to the slope of the production
possibilities frontier. With public goods, this is not the outcome that you want.

An alternative approach that helps to explain how contributions are determined
and why these contributions are not pareto optimal is to try find the best
choice for consumer $1$ to make for all the different possible contribution
levels that consumer $2$ might choose. This approach is more common in game
theory and involves the construction of something called the \emph{best reply
function}. The best reply functions are put together to understand the final outcome.

In Figure 2 the various \emph{consumption choices} that player $2$ can make
are given along the bottom axis. Each such choice implies a contribution to
the production of the public good - just take the difference between the
consumption level and the endowment to find it (this figure would look a bit
different if player $2$'s \emph{contributions} to production of the public
good were listed on the bottom axis - that approach is more common).%

%TCIMACRO{\FRAME{ftbpF}{3.5414in}{2.399in}{0pt}{}{}{public_goods_fig2.eps}%
%{\special{ language "Scientific Word";  type "GRAPHIC";
%maintain-aspect-ratio TRUE;  display "USEDEF";  valid_file "F";
%width 3.5414in;  height 2.399in;  depth 0pt;  original-width 3.3909in;
%original-height 2.2883in;  cropleft "0";  croptop "1";  cropright "1";
%cropbottom "0";  filename 'public_goods_fig2.eps';file-properties "XNPEU";}}}%
%BeginExpansion
\begin{figure}
[ptb]
\begin{center}
\includegraphics[
height=2.399in,
width=3.5414in
]%
{public_goods_fig2.eps}%
\end{center}
\end{figure}
%EndExpansion
The green lines represent iso-utility curves for consumer $1$. They are
solutions to equations of the form%
\[
u_{1}\left(  x_{1},f\left(  \omega_{1}+\omega_{2}-x_{1}-x_{2}\right)  \right)
=K
\]
where $K$ is some constant. Consumer $1$ achieves higher utility (holding is
own consumption level constant) the \emph{lower} is the consumption level of
consumer $2$. Lower consumption by consumer $2$ means that consumer $2$ is
contributing more of his or her endowment to production of the public good.

In the Nash equilibrium, consumer $1$ forms some belief about the consumption
level that consumer $2$ will pick. Suppose for the moment, that he believes
that consumer $2$ will choose consumption $x^{\prime\prime}$. Then he can
attain any $\left(  x_{1},x_{2}\right)  $ combination that lies on the
vertical line through $x^{\prime\prime}$. The best such point is the one that
lies on the highest iso-utility curve. That is the one through the point
$\left(  R_{1}\left[  x^{\prime\prime}\right]  ,x^{\prime\prime}\right)  $
where an iso-utility curve is just tangent to this vertical line. (If he were
to lower his planned consumption by moving down this vertical line, he would
end up on a lower iso-utility curve like the one that lies just to the right
of the point $\left(  R_{1}\left[  x^{\prime\prime}\right]  ,x^{\prime\prime
}\right)  $.

There would be a different best choice like this for every different choice
that consumer $2$ makes. The picture shows the corresponding tangencies at
point $x^{\prime}$ and $x^{\prime\prime\prime}$. If you joined all these best
choices together they would form a line (not necessarily straight as it is in
the picture) called consumer $1$'s reaction function. This is the line
$R_{1}R_{1}$ in the picture. It explains what consumer $1$ would choose to do
for every possible different belief that he might have about the consumption
choice of consumer $2$.%

%TCIMACRO{\FRAME{ftbpF}{3.2949in}{2.412in}{0pt}{}{}{public_goods_fig3.eps}%
%{\special{ language "Scientific Word";  type "GRAPHIC";
%maintain-aspect-ratio TRUE;  display "USEDEF";  valid_file "F";
%width 3.2949in;  height 2.412in;  depth 0pt;  original-width 3.1531in;
%original-height 2.3004in;  cropleft "0";  croptop "1";  cropright "1";
%cropbottom "0";  filename 'public_goods_fig3.eps';file-properties "XNPEU";}}}%
%BeginExpansion
\begin{figure}
[ptb]
\begin{center}
\includegraphics[
height=2.412in,
width=3.2949in
]%
{public_goods_fig3.eps}%
\end{center}
\end{figure}
%EndExpansion

Doing the same exercise for consumer $2$ yields a similar curve, which is
drawn in Figure $3$ as $R_{2}R_{2}$. The point where these two curves
intersect is the Nash equilibrium. Each consumer chooses his best consumption
given what he expects the other consumer to choose, and as it turns out the
other consumer always does exactly what he expects him to do.

When you try to construct the iso utility curves for person $2$, he will
choose one that is tangent to the flat line through $x_{1}^{\ast}$ since he
expects consumer $1$'s consumption choice to $x_{1}^{\ast}$ no matter what he
does. This means that the iso utility curves for the two consumers must cut
through each other as shown in the diagram. There must be a point like $E$
where both of the consumers would be better off if they could jointly agree to
move there. That would involve each of them reducing their own consumption of
the private good, and using that to increase production of the public good.

\section{Resolving the Public Good Problem}

There is a rather surprising way to resolve this problem. The entire
difficulty with public goods arises because a consumer who raises his or her
contribution to the public good increases the utility of the other trader. To
get consumers to choose the right amounts something has to be done to
'internalize' this externality. Here is one way to do it.

First declare that there are actually 3 goods, $y_{1}$, $y_{2}$ and $x$. The
first is public good for person $1$, the second public good for person $2$ and
the third the private good. All production of these three goods will be
undertaken by a single profit maximizing firm whose production possibilities
frontier is just%
\[
\left\{  \left(  y_{1},y_{2},x\right)  :y_{1}=y_{2}=f\left(  \omega_{1}%
+\omega_{2}-x\right)  \right\}
\]
All endowments are owned by the firm, but consumer $1$ owns the share
$\omega_{1}/\left(  \omega_{1}+\omega_{2}\right)  $ of the firm and will
receive that share of its profits. Consumer $2$ will own the complementary
share $\omega_{2}/\left(  \omega_{1}+\omega_{2}\right)  $. We will make one
big assumption, which is that this single firm is a price taker.

Consumer $1$ only cares about his consumption of the private good and his
consumption of good $1$, person $2$ only cares about his consumption of the
private good and good $2$. Person $1$ doesn't care how much $y_{2}$ person $2$
consumes, so there are no externalities in consumption. The firm
'internalizes' all the externalities associated with the public good. The
physical connection between $y_{1}$ and $y_{2}$ is simply a part of its
production process. So there are no production externalities. So all we need
to do is to find the Walrasian equilibrium of this economy with production and
that will give us a pareto optimal allocation by the first welfare theorem
that we studied last week 	REFERENCE ``PROFIT MAXIMIZATION'' CHAPTER.

The solution is given in Figure 4.%

%TCIMACRO{\FRAME{ftbpF}{4.9995in}{2.3506in}{0pt}{}{}{public_goods_fig4.eps}%
%{\special{ language "Scientific Word";  type "GRAPHIC";
%maintain-aspect-ratio TRUE;  display "USEDEF";  valid_file "F";
%width 4.9995in;  height 2.3506in;  depth 0pt;  original-width 4.9441in;
%original-height 2.3108in;  cropleft "0";  croptop "1";  cropright "1";
%cropbottom "0";  filename 'public_goods_fig4.eps';file-properties "XNPEU";}}}%
%BeginExpansion
\begin{figure}
[ptb]
\begin{center}
\includegraphics[
height=2.3506in,
width=4.9995in
]%
{public_goods_fig4.eps}%
\end{center}
\end{figure}
%EndExpansion

Notice that when the firm increases production of the public good, it receives
revenue twice on each unit it produces. Consumer $1$ pays $p_{1}$ for that
unit, but consumer $2$ also pays $p_{2}$ for it. So the iso profit curve for
the profit maximizing firm that must be tangent to the production
possibilities frontier has slope $\frac{1}{p_{1}+p_{2}}$. The firm earns its
profits for its production decision then distributes these profits to its
owners. Consumer $1$ receives income
\[
\frac{\omega_{1}}{\omega_{1}+\omega_{2}}\left(  p_{1}y^{\ast}+p_{2}y^{\ast
}+x_{1}^{\ast}+x_{2}^{\ast}\right)
\]
which is labelled on the horizontal axis in Figure 4 . Consumer $2$ receives%
\[
\frac{\omega_{2}}{\omega_{1}+\omega_{2}}\left(  p_{1}y^{\ast}+p_{2}y^{\ast
}+x_{1}^{\ast}+x_{2}^{\ast}\right)
\]
which is the point where $2$'s budget line intersects the horizontal axis
(that has not been labelled in the figure to keep things simpler).

Consumer $1$ now faces a budget line (the blue line in the picture) along
which he chooses his best consumption bundle. Notice that since consumer $1$
only buys good $y_{1}$ at price $p_{1}$ the slope of this budget line is
$\frac{1}{p_{1}}$ \emph{not }$\frac{1}{p_{1}+p_{2}}$. So in this equilibrium,
consumers marginal rates of substitution will be different from the firms
marginal rate of substitution and generally different from each other.

The market clearing conditions are twofold, first consumers must both choose
to purchase the common level of output of the public good that has been
offered by the firm. Second the sum of the private good demand of each
consumer must be equal to the total amount of the private good that the firm
has chosen to produce.

The prices that support this outcome are often know as Lindahl prices. The
reciprocal of the slope of the production possibilities frontier is the
marginal cost of producing one extra unit of the public good (expressed in
terms of units of the private good). Since the iso profit line must be tangent
to the iso profit curve, this marginal cost is equal to $p_{1}+p_{2}$. The
reciprocal of the slopes of the consumers indifference curves are equal to
their marginal willingness to pay for the public good (again expressed in
terms of the private good). Since the indifference curves are tangent to the
individual budget lines, these willingnesses to pay are $p_{1}$ and $p_{2}$
respectively. So the Lindahl prices ensure that the marginal cost of producing
the public good is exactly equal to the sum of the two consumers willingness
to pay.
\end{document}