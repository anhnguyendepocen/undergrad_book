%% LyX 2.1.4 created this file.  For more info, see http://www.lyx.org/.
%% Do not edit unless you really know what you are doing.
\documentclass[oneside,english]{amsart}
\usepackage[T1]{fontenc}
\usepackage[latin9]{inputenc}
\usepackage{color}
\usepackage{amsthm}

\makeatletter

%%%%%%%%%%%%%%%%%%%%%%%%%%%%%% LyX specific LaTeX commands.
%% Because html converters don't know tabularnewline
\providecommand{\tabularnewline}{\\}

%%%%%%%%%%%%%%%%%%%%%%%%%%%%%% Textclass specific LaTeX commands.
\numberwithin{equation}{section}
\numberwithin{figure}{section}
\theoremstyle{plain}
\newtheorem{thm}{\protect\theoremname}
  \theoremstyle{definition}
  \newtheorem{defn}[thm]{\protect\definitionname}

\makeatother

\usepackage{babel}
  \providecommand{\definitionname}{Definition}
\providecommand{\theoremname}{Theorem}

\begin{document}

\title{Deferred Acceptance}

\maketitle
The point of this note is to take you through the calculations involved
when using the deferred acceptance algorithm. Please read the papers
associated with the links on the web page for motivation and applications.

We'll use the example of college placement since you might soon be
involved in that, and try to work up to something realistic. To begin
we'll just start with three postgraduate programs or schools, call
them UBC, SFU, and UoT. When we want to refer to the set of schools
we can just write $Schools.$ We'll also suppose that there are exactly
three students. Lets just call them student 1, 2 and 3 eventually
we will have a lot of them, and it won't make sense anymore to give
them names. We'll just write $Students$ to refer to the set of students.

In this introductory example, lets just suppose that each school can
only admit one student. Of course, each of the students can only go
to one school, so we want to place the students in some reasonable
way. What we mean by reasonable has to be made precise. To do this
we need to know the schools and students preferences. Lets write the
schools' preferences in the following way
\[
SFU\rightarrow1\succ2\succ3
\]
\[
UBC\rightarrow1\succ3\succ2
\]
\[
UOT\rightarrow2\succ3\succ1
\]
What these bits of notation mean is that SFU wants student 1 the most.
If student 1 doesn't come to SFU then they prefer to have student
2, while student 3 is their least favorite student. UBC also really
wants student 1, but has a different ranking of students 2 and 3.
Finally UOT really wants student 2, but if they can't get her, they
prefer student 3 to student 1.

We can do the same thing for the students
\[
1\rightarrow UBC\succ SFU\succ UOT
\]
\[
2\rightarrow UBC\succ UOT\succ SFU
\]
\[
3\rightarrow UBC\succ SFU\succ UOT
\]
(the marketing person told me I had to write the preferences that
way). All three students really want to get into UBC. Otherwise, students
1 and 3 want SFU if they are unable to get into UBC. Student 2 prefers
UOT if she can't get into UBC.

Now we want to 'place' the students in schools, which means that we
want to define a \emph{match $m:Students\rightarrow Schools$. }Then
we would write $m\left(1\right)=UBC$ to indicate that student 1 is
to be placed in UBC. So far, a match needs to be one to one and onto.

A little visual aid might help you think about the matching. Lets
start by drawing a table that looks like a normal form game (but isn't).

\begin{center}
\begin{tabular}{|c|c|c|c|}
\hline 
 & 1 & 2 & 3\tabularnewline
\hline 
\hline 
SFU & 1,2 & 2,3 & 3,2\tabularnewline
\hline 
UBC & 1,1 & 3,1 & 2,1\tabularnewline
\hline 
UOT & 3,3 & 1,2 & 2,3\tabularnewline
\hline 
\end{tabular}
\par\end{center}

The first element of each cell is the \emph{rank }that the university
in the corresponding row assigns to the student in the corresponding
column. As above, if you look across the SFU row at the top, student
1 is most preferred by UBC, so it has rank 1. Similarly, student 2
has rank 2, etc. 

As you read down the column, you can read how each of the students
ranks the various universities by looking at the second element in
each cell. Reading down column 1, UBC has rank 1, SFU rank 2 and UOT
has rank 3. This represents the preferences in a compact way. 

We can represent a matching by placing three stars in the table in
such a way that each row has exactly 1 start in it, and each column
has exactly one star in it. For example, here is a matching that puts
student 1 in SFU, student 2 in UBC and student 3 in UOT:

\begin{center}
\begin{tabular}{|c|c|c|c|}
\hline 
 & 1 & 2 & 3\tabularnewline
\hline 
\hline 
SFU & 1,2{*} & 2,3 & 3,2\tabularnewline
\hline 
UBC & 1,1 & 3,1{*} & 2,1\tabularnewline
\hline 
UOT & 3,3 & 1,2 & 2,3{*}\tabularnewline
\hline 
\end{tabular}
\par\end{center}

With this simple device, we can start to think about how the students
and schools \emph{should} be matched. All of the students want to
be in UBC, which isn't possible since each school can only accept
one student. One obvious criteria that a good match should satisfy
is that it be \emph{pareto optimal. }A match would be pareto dominated
if there were some other match that made at least one school or student
strictly better off without hurting anyone else. 

Under this criteria, the matching $m\left(3\right)=UBC$, $m\left(2\right)=SFU$,
and $m\left(1\right)=UOT$ would be something we would definitely
call a bad matching. It looks like this:

\begin{center}
\begin{tabular}{|c|c|c|c|}
\hline 
 & 1 & 2 & 3\tabularnewline
\hline 
\hline 
SFU & 1,2 & 2,3{*} & 3,2\tabularnewline
\hline 
UBC & 1,1 & 3,1 & 2,1{*}\tabularnewline
\hline 
UOT & 3,3{*} & 1,2 & 2,3\tabularnewline
\hline 
\end{tabular}
\par\end{center}

Lets see how to improve on it by trying to make some of the schools
and students better off without hurting the others. Student 3 has
her favorite program, so if we make any change that moves her, she
will be worse off. So lets focus on students 1 and 2 who both get
their least favorite school. If we switch them around, they would
both be made strictly better off. All we need to check is that the
schools aren't hurt by this change. Since UOT prefers student 2 to
student 1, $UOT$ would be made better off. Similarly, since $SFU$
prefers student 1 to student 2, SFU is also made better off.

This new matching $m\left(3\right)=UBC$, $m\left(1\right)=SFU$,
and $m\left(2\right)=UOT$ is pareto optimal (if we try to move student
1 to UBC we'll hurt player 3). However, this points out that there
is still a problem with this new matching because UBC would strictly
prefer student 1 to student 3, and student 1 strictly prefers UBC
to SFU where they are now matched. If we somehow come up with a game
or a mechanism or a market that generates the matching $m\left(3\right)=UBC$,
$m\left(1\right)=SFU$, and $m\left(2\right)=UOT$, then UBC will
want to negotiate a side deal with student 1. We would then say that
our mechanism is not \emph{stable }so our market would \emph{unravel.}

Formally, a match $m$ is said to be \emph{stable }if for any student
$i$ and university $j$, if student $i$ prefers university $j$
to university $m\left(i\right)$, then university $m\left(i\right)$
prefers the student who it gets in $m$ (that would be $m^{-1}\left(j\right)$)
to student $i$. If that is true, then any attempt to unravel the
matching process by some kind of offer outside the match will be rejected.

It is pretty tedious checking all this stuff as you can see. Especially
if there were thousands of students and many different schools involved.
You might wonder whether we could find an easier way to find a matching
that is not just pareto optimal, but also stable. This is where the
deferred acceptance idea comes into play. 


\subsection*{Algorithms}

An algorithm is a method of systematically checking data to find something
we are looking for. You have seen one algorithm so far - we talked
about checking normal form games to find all the Nash equilibria.
That algorithm was really a very poor one - for example, to check
for pure strategy equilibria we need to check every cell in the payoff
matrix to see if there were profitable deviations. 

Now instead of looking for a Nash equilibrium, we are looking for
a stable matching. This is pretty complicated because we seem to have
to check lots of pairings to check for stability. Fortunately there
is a faster way to do this. 

This is how the story goes - first we ask each of the students to
\emph{propose }to their favorite university. This doesn't seem to
help much because they all like UBC. Nonetheless, lets suppose each
of them makes such a proposal. UBC now responds by picking its favorite
applicant - student 1 in our example. UBC \emph{tentatively }accepts
student 1's proposal. The reason for the word 'tentative' won't be
clear in this example, but the idea we have in mind is that UBC might
later get a better proposal and we want to allow UBC to accept this
proposal if it ever comes along. 

We could visualize these proposals in the following simple way:

\begin{center}
\begin{tabular}{|c|c|c|c|}
\hline 
 & 1 & 2 & 3\tabularnewline
\hline 
\hline 
SFU & 1,2 & 2,3 & 3,2\tabularnewline
\hline 
UBC & 1,1{*} & 3,1{*} & 2,1{*}\tabularnewline
\hline 
UOT & 3,3 & 1,2 & 2,3\tabularnewline
\hline 
\end{tabular}
\par\end{center}

This isn't a feasible matching because UBC's row has three stars in
it, and it is only allowed 1. To indicate that UBC has rejected the
proposals by 2 and 3 we could write

\begin{center}
\begin{tabular}{|c|c|c|c|}
\hline 
 & 1 & 2 & 3\tabularnewline
\hline 
\hline 
SFU & 1,2 & 2,3 & 3,2\tabularnewline
\hline 
UBC & 1,1{*} & \textcolor{red}{3,1{*}} & \textcolor{red}{2,1{*}}\tabularnewline
\hline 
UOT & 3,3 & 1,2 & 2,3\tabularnewline
\hline 
\end{tabular}
\par\end{center}

At this stage UBC has tentatively accepted student 1's proposal, students
2 and 3 are still left without places. We have actually learned quite
a bit from this. Students 2 and 3 know that UBC has just tentatively
accepted player 1's proposal. Yet they also know that if UBC ever
throws that proposal out in future, it will be for a student that
it strictly prefers to student 1. Then by transitivity, this student
will be preferred to students 2 and 3, in other words, there is no
point in students 2 and 3 bothering to propose to UBC again.

\begin{center}
\begin{tabular}{|c|c|c|c|}
\hline 
 & 1 & 2 & 3\tabularnewline
\hline 
\hline 
SFU & 1,2 & 2,3 & 3,2{*}\tabularnewline
\hline 
UBC & 1,1{*} & \textcolor{red}{3,1{*}} & \textcolor{red}{2,1{*}}\tabularnewline
\hline 
UOT & 3,3 & 1,2{*} & 2,3\tabularnewline
\hline 
\end{tabular}
\par\end{center}

So in stage 2, we want students 2 and 3 to make proposals to schools
who have not yet rejected them - 2 will propose to UOT, 3 will propose
to SFU. Since the universities have only one proposal, they accept
that proposal, and we are finished. This stable matching is $m\left(1\right)=UBC$,
$m\left(2\right)=UOT$, $m\left(3\right)=SFU$ 

Unlike our previous effort, this matching won't unravel. SFU does
badly in a way, since they get their least preferred applicant. Yet
if SFU approaches student 1 or 2 who they prefer, they will be rejected,
because each of them is already matched to a university that they
prefer to SFU.

This, in a very simple form, is how the deferred acceptance \emph{algorithm
}works.


\subsection{Implementing the Deferred Acceptance Solution}

Now we are starting to get somewhere. We have something that looks
like data, and a process for turning the data into an outcome that
we like. The last step is to turn this process and algorithm into
something that works in practice. To do this, we need to find some
way to convince market participants to voluntarily give us the data
we want about preferences. 

One way we could do it is to have the students physically apply to
each of the schools just as we suggested above. Many matching 'markets'
work like this. In fact, the process of applying to firms who then
choose which applicants to hire is the way job markets would be described
in most economics applications. 

The market with which I am most familiar is the market for junior
academic economists. The applicants are economics students graduating
with Ph'd degrees. The firms are Universities, Governments, and Consulting
firms throughout the world. The market runs through much of the year,
but the primary hiring period begins in October when Universities
place job ads. In 2014 there were a little more than 4000 applicants
registered at econjobmarket.org, one of the main advertising and recruiting
sites. The number of academic openings is somewhere between 1200 and
1600. Junior economists apply to schools they are interested in. The
schools review applications, interview applicants at a centralized
meeting in January, then make offers to the applicants they like.
The applicants, many of whom receive multiple offers, choose the offer
that they like.

This might give you some idea about the potential size of a market,
and how complex it might be to find some kind of stable matching.
It mostly illustrates that it won't be feasible to find a good solution
by scribbling on pieces of paper, as we did above. The market for
junior economists has some complexities that we'll discuss later. 

In 2013, the Vancouver School Board implemented a matching program
for children in special kindergarten programs. There were four programs,
French Immersion, Mandarin Immersion, Montessori, and Fine Arts, spread
over 19 schools. In the year prior to their program, about 1400 students
participated. The school board discussed the possibility of matching
students using deferred acceptance, but later changed their mind and
used a simple computerized lottery using only parents' first choice.
This system replaced one in which parents physically traveled to as
many schools as they wanted and submitted an application. Schools
then ran their own lotteries.

Neither of these examples actually use the deferred acceptance algorithm
that we will discuss below. In the school board case, the board was
concerned about the difficulty parents might have expressing their
preferences over schools. In the the academic job market, universities
don't have clear preferences at the start of the process and need
time to develop them. There is no hope of implementing a pareto optimal
and stable matching if you don't know participants preferences. The
Vancouver school board example is an interesting case in point. The
way their current matching program works is that each parent lists
three schools in order or their preference. The actual lottery only
makes use of the first choice. Each child is entered into a draw for
their parents first choice program. If they win the draw, that is
great. If they don't, they are put on a waiting list for that school
in case another family who did win the lottery decides to decline.
Since parents who listed a school as their first choice are always
put on a waiting list, the second and third choice are basically not
attainable. If parents are very anxious to get their children into
a special program, they need to choose the program where they think
they are most likely to get in. This may or may not be the one they
most prefer for their child.

This is one of the reasons that deferred acceptance is advantageous.
The side who make proposals (parents in the case of the Vancouver
School Board, students in the college application example) have no
reason to mis-represent their preferences the way they do with the
Vancouver School Board example. One of the things that is important
in market design is which side should make proposals.

We could alter our procedure above and have the schools make proposals.
First, UBC and SFU could both send proposals, or acceptance offers
to student 1 whom they most prefer. UOT sends an acceptance letter
to student 2 since they like her best. Student 1 tentatively accepts
the offer by UBC and rejects SFU, while student 2 accepts the offer
from UOT. SFU sends an offer to student 2, but student 2 rejects it
because he likes UOT better, finally SFU makes the offer to student
3, who accepts it and we are done. Same matching as when students
propose.

As long as student 1 can hold onto his offer until he learns whether
or not he will get a better one, this procedure works pretty well.
One of the things that can make this procedure fail is something called
an 'exploding' offer. UBC might send an acceptance to student 1, but
require that student 1 accept it almost immediately - in particular,
before student 1 learns whether he might get offers from other schools.
This doesn't make any difference here since student 1 won't get any
better offers anyway. Yet this is generally a major problem for matching
markets.

This brings us to modern mechanism design since computers give us
a way to do all this almost instantaneously. Suppose we organize this
market the following way - each student simply goes to a web page
and writes down her preferences exactly as we did at the top of this
note. In other words, student 1 simply looks at a list consisting
of the items UBC,SFU and UOT and shifts the elements of the list up
and down until her favorite school is at the top, second favorite
in the middle, etc. She clicks on submit, and a computer now knows
her preferences in the same way we did when we discussed this above.
The other students and the universities do exactly the same thing.
Now the computer simply goes through the steps we describe above,
and finds the matching $m\left(1\right)=UBC$, $m\left(2\right)=UOT$,
$m\left(3\right)=SFU$ using either the student proposing or school
proposing deferred acceptance method.

Now we are at the point we wanted since the students and schools are
basically involved in a simple game. The actions available to the
students are the various preferences they could provide on the web
page, i.e $UBC\succ SFU\succ UOT$, $UBC\succ UOT\succ SFU$, etc
(there are $3!$ or 6 of them in all). Given the strategies of the
other students and schools, each such preference results in a matching.
For example, if the others are expected to announce their preference
as we described them above, we could calculate what student 1 could
achieve for each possible preference she might express. It looks like
this

\begin{center}
\begin{tabular}{|c|c|}
\hline 
$UBC\succ SFU\succ UOT$ & $UBC$\tabularnewline
\hline 
\hline 
$UBC\succ UOT\succ SFU$ & $UBC$\tabularnewline
\hline 
$SFU\succ UBC\succ UOT$ & $SFU$\tabularnewline
\hline 
$SFU\succ UOT\succ UBC$ & $SFU$\tabularnewline
\hline 
$UOT\succ SFU\succ UBC$ & $SFU$\tabularnewline
\hline 
$UOT\succ UBC\succ SFU$ & $SFU$\tabularnewline
\hline 
\end{tabular}
\par\end{center}

Lets go through the second to last line to see how the table works.
The first entry is $UOT\succ SFU\succ UBC$ which means that student
1 submits the preference $UOT\succ SFU\succ UBC$ to the computer
that will then use the student proposing version of the algorithm
to find a matching. If player 1 submits this preference, then the
computer will submit applications to UBC for students 2 and 3 and
submit an application to UOT for student 1. At that point, the computer
can see that UBC prefers student 3 to student 2, while UOT has only
an application from student 1, so the computer will reject student
2 from UBC and go on to the next round. 

In the second round, the computer thinks that student 2 prefers UOT
to SFU, so it will submit an application to UOT for student 2, and
use this to reject the application by student 1. At that point students
2 and three have been matched, so in the final round, the computer
just submits an application to SFU for student 1. This is what the
rightmost column says - it gives the school that student 1 will be
matched to if he submits that preference to the computer. Since student
1 prefers UBC to SFU, it isn't in his interest to submit such a preference.
It is a best reply for student 1 to submit his true preference to
the computer if he expects the others to submit preferences as we
described at the beginning of this note.

It isn't always in everyone's interest to submit their preferences
truthfully. For example, suppose we look at a simpler problem with
only SFU and UBC and only two students, 1 and 2. We'll add the option
of not going to university and suppose that preferences look like
this
\[
UBC\rightarrow1\succ2
\]
\[
SFU\rightarrow2\succ1
\]
while the students' preferences are
\[
1\rightarrow SFU\succ UBC\succ OUT
\]
and
\[
2\rightarrow UBC\succ SFU\succ OUT
\]


The student proposing algorithm has 1 applying to SFU and 2 applying
to UBC. That is it, there are no more proposals. So $m\left(1\right)=SFU,$
$m\left(2\right)=UBC$. The school proposing algorithm has UBC proposing
to 1 and SFU proposing to 2, so the matching is $m\left(1\right)=UBC$,
$m\left(2\right)=SFU$. Notice that both the schools are better off
in the school proposing algorithm than they are in the student proposing
algorithm.

Now notice that something strange happens in this example. Suppose
that in the school proposing version, that student 1 simply lies and
says that if he can't get into school, he would rather not to go school
at all. In other words, he or she submits the preference $1\rightarrow SFU\succ OUT\succ UBC$.
Lets run the school proposing version of the algorithm - UBC proposes
to 1 and SFU proposes to 2. At this point, our computerized algorithm
decides that 1 should reject UBC's proposal because it thinks he would
rather not go to school at all. Since its offer has been rejected,
UBC now proposes to its second favorite choice, student 1, who drops
the SFU offer and goes to UBC. SFU now has to make an offer to student
1, who accept it. By lying student 1 makes himself strictly better
off, but only in the school proposing version of the algorithm. 

Here are the two remarkable facts about this computerized procedure:
\begin{enumerate}
\item The student proposing version of the deferred acceptance algorithm
(which we just described) will always lead (in a finite number of
steps) to a matching that is stable given the submitted preferences:
and
\item It is always in the interests of the students to submit their preferences
truthfully no matter what preferences they think the others are going
to submit.
\end{enumerate}
It is hard to emphasize how useful these observations are. First,
the process of sending letters and acceptances and agonizingly waiting
to hear if a better offer will come is accomplished in the blink of
an eye by the computer. Speed is not the only issue, all the time
spent handing the communication is also eliminated. 

Second, students send their preferences to the computer which can
record them for future use. Students never want to lie about their
preferences, this isn't a happiness survey, it is in the interests
of the students to reveal they way they would make choices on their
own. Now instead of relying on universities web pages, which almost
always make the claim that they are one of the top universities in
the world, you can just look at the preferences that students submitted
and see for yourself. Recall that even if you find that one university
is very popular, there is no reason for you to pretend that you like
some less popular university in the hopes that you will get in there
once you are refused at the popular university. The computer already
takes care of that.

Just so you can see the kind or reasoning used in the literature on
this topic, we can go through a proof of the first of these two facts.
As always, you should be a skeptic. Just because the algorithm produces
a stable matching with the examples that I used above doesn't mean
it always will. Anyway, why exactly does the algorithm work in the
the examples above. When you think of it, it is pretty hard to check.
To do so, one really should take every pair of universities and students
and ask whether the student chosen would prefer to be in this alternative
university instead of being in the match produced by the algorithm,
while at the same time the university would prefer to admit the student
(possibly after kicking out someone else) to the group with whom they
are matched during the algorithm.

We need a bit of formality to do this. Suppose there are $n$ students
each indicated by an index $j$. So $j\in n$ just means that $j$
is the name of a student. At the same time, suppose that there are
$m$ universities (probably $m$ is much smaller than $n$). The notation
$i\in m$ just means that $i$ is the name of a university. This convention
is adopted in what follows, $i$, and its variants like $i^{\prime}$
always refer to a university, $j$ to a student. Lets refer to a matching
by the Greek letter $\mu$. Then if we put the name of a student into
$\mu$, the result will be the name of the university that the student
has been placed in. For example, in the example we started with at
the beginning of this note, $\mu\left(3\right)=UBC$. If we put the
name of a university in the matching function, for example, $\mu\left(i\right)$,
we get a list of all the students who were placed in the university
by our matching procedure. When we need it, we'll let $k_{i}$ be
the capacity of university $i$, i.e., the maximum number of students
that $i$ can admit.

Second, we need an assumption about university preferences. Let $S$
be any collection of students who have been admitted to some university
$i$, and that $i$ has space for some more students. Let $j$ and
$j^{\prime}$ be two students who the university has not yet admitted.
\begin{defn}
University $i$ has responsive preferences if $S\cup\left\{ j\right\} \succ_{i}S\cup\left\{ j^{\prime}\right\} \implies\forall S^{\prime};S^{\prime}\cup\left\{ j\right\} \succ_{i}S^{\prime}\cup\left\{ j^{\prime}\right\} $.

What this definition says is that whether or not a university likes
one student more than another doesn't depend on the set of students
the university has already admitted (or on the set it expects to admit).
An example is a university that ranks students according to their
GPA or SAT scores.\end{defn}
\begin{thm}
If universities have responsive preferences, and students and universities
provide their true preferences, then the deferred acceptance algorithm
will always produce a stable matching.\end{thm}
\begin{proof}
The matching is supposed to be stable in this theorem, so suppose
the $\mu^{\prime}$ is a matching produced by the deferred acceptance
algorithm which isn't stable. We'll just prove that if the matching
isn't stable, then it couldn't have been produced by deferred acceptance.

First, what does it mean to say that the matching isn't stable. If
that is true, then there is a student, say $j$ and a university,
say $i$, such that $j$ would strictly prefer to be in university
$i$ that in university $\mu^{\prime}\left(j\right)$ where they were
placed by deferred acceptance, while at the same time, university
$i$ would either like to add student $j$ to the set of students
it has already accepted (if it has space), or would like to replace
one of its existing students with student $j$ (if it is full).

Now the rules of the deferred acceptance algorithm are very strict
- at each step the student must send an application to the university
he or she most prefers among the group of universities that have not
yet rejected one of their applications. Since $j$ must have sent
an application to $\mu^{\prime}\left(j\right)$, but prefers $i$
to $\mu^{\prime}\left(j\right)$, it must have been the case that
$j$ sent an application to $i$ which was rejected. That is just
the way the algorithm works, if $j$ hadn't received a rejection from
$i$, the program would not have sent an application to $\mu^{\prime}\left(j\right)$
and would have sent it to $i$ instead.

The next question is why $i$ would have rejected $j$'s application.
The only time the program will allow the university to reject an application
is when it has already received applications from a group of students
all of whom are preferred to applicant $j$.\footnote{We don't really need to worry about university $i$ rejecting student
$j$ because their GPA isn't high enough - if that were true the university
would never have wanted to admit $j$ after the matching either.} So the university liked its existing applicants better when $j$
applied. After that, the algorithm would continue to process new applications
for the university, some students would be admitted, and others dropped,
some would be rejected. No matter how long this process continued
after $j$'s application, the university would only have admitted
new students that it preferred to one of their existing acceptances.
Since we now know that the existing applicant must have been preferred
to $j$, the new applicants who are admitted must also be preferred
to $j$ (by responsive preferences), so it is impossible for the university
to want to admit $j$ after the matching is finished. By this Sherlockian
deductive process we can conclude that there cannot be a matching
produced by the deferred acceptance algorithm that is unstable.
\end{proof}
Notice a few things about this argument. First, we need the university
to have responsive preferences. For example, the Stanford MBA program
might prefer a student with an engineering background to a student
with an English degree if it expects lots of applications from good
humanities students. Yet it might find that near the end of the algorithm
that it hasn't received the humanities applications it expected. Then
it might decide to change its mind. This would be a violation of responsive
preferences, and you can see that our argument above would fail in
this case (which is why we added the assumption).

Of course, the second thing to consider here is that the algorithm
uses the preferences that the university submitted at the very start.
If these aren't the university's real preferences, then it might well
want to admit the student once it finds out that its attempt to manipulate
the outcome hasn't worked the way it wanted.

I'll stick with the student proposing version of the algorithm in
what follows, and just note it isn't always in the interests of the
universities to submit their preferences truthfully. Often this doesn't
matter that much. Universities have moved away from screening students
to graduating as many students as they can. So typically the constraints
from the universities point of view are just capacity. We'll explicitly
come back to deal with this since it is an important consideration
in a lot of matching problems. So lets just try something slightly
more complicated.


\subsection{A slightly more complicated problem:}

Lets make things slightly more complicated by adding a few more students,
another university, and another option, The additional option will
be not going to university at all. Here are the preferences. At this
point you can imagine that these are preferences that were actually
submitted through the web page. The problem is just to do the matching.
Here are the preferences written in two different ways:
\[
SFU\rightarrow1\succ4\succ2\succ3
\]
\[
UBC\rightarrow1\succ3\succ4\succ2
\]
\[
UOT\rightarrow2\succ3\succ1\succ4
\]


\[
UFV\rightarrow2\succ3\succ4\succ1
\]


\[
1\rightarrow UBC\succ SFU\succ UOT\succ UFV\succ OUT
\]
\[
2\rightarrow UBC\succ UOT\succ SFU\succ UFV\succ OUT
\]
\[
3\rightarrow UBC\succ SFU\succ UOT\succ UFV\succ OUT
\]


\[
4\rightarrow UBC\succ UOT\succ SFU\succ OUT\succ UFV
\]


\begin{center}
\begin{tabular}{|c|c|c|c|c|}
\hline 
 & 1 & 2 & 3 & 4\tabularnewline
\hline 
\hline 
UBC & 1,1 & 4,1 & 2,1 & 3,1\tabularnewline
\hline 
SFU & 1,2 & 3,3 & 4,2 & 2,3\tabularnewline
\hline 
UFV & 1,4 & 1,4 & 2,4 & 3,5\tabularnewline
\hline 
UOT & 3,3 & 1,2 & 2,3 & 4,2\tabularnewline
\hline 
OUT & -,5 & -,5 & -,5 & -,4\tabularnewline
\hline 
\end{tabular}
\par\end{center}

We've added a fourth university (if you like name UFV means University
of the Fraser Valley). There is also a fourth student who only wants
to go to UBC, SFU or UOT, but is not interested in UFV.

Lets find a stable matching using the algorithm:
\begin{enumerate}
\item All students propose to UBC, which tentatively accepts student 1 and
rejects the other proposals;
\item Students 2 and 4 propose to UOT, while student 3 proposes to SFU,
SFU tentatively accepts student 3, UOT tentatively accepts student
2 who it prefers;
\item Student 4, who has been rejected by UBC and UOT, now proposes to his
remaining favorite SFU. SFU prefers 4 to their current acceptance,
3, so they throw 3 out and accept 4;
\item Student 3 has been rejected by UBC and now SFU, so he proposed to
his next favorite, UOT who has currently accepted 2 who they prefer,
student 3's application is rejected;
\item Finally, student 3 proposes to UFV who accepts him.
\end{enumerate}
This gives us
\[
m\left(1\right)=UBC,m\left(2\right)=UOT,m\left(3\right)=UFV,m\left(4\right)=SFU.
\]


Now in pictures:

\begin{center}
\begin{tabular}{|c|c|c|c|c|}
\hline 
 & 1 & 2 & 3 & 4\tabularnewline
\hline 
\hline 
UBC & 1,1{*} & 4,1{*} & 2,1{*} & 3,1{*}\tabularnewline
\hline 
SFU & 1,2 & 3,3 & 4,2 & 2,3\tabularnewline
\hline 
UFV & 1,4 & 1,4 & 2,4 & 3,5\tabularnewline
\hline 
UOT & 3,3 & 1,2 & 2,3 & 4,2\tabularnewline
\hline 
OUT & -,5 & -,5 & -,5 & -,4\tabularnewline
\hline 
\end{tabular}
\par\end{center}

\begin{center}
\begin{tabular}{|c|c|c|c|c|}
\hline 
 & 1 & 2 & 3 & 4\tabularnewline
\hline 
\hline 
UBC & 1,1{*} & \textcolor{red}{4,1{*}} & \textcolor{red}{2,1{*}} & \textcolor{red}{3,1{*}}\tabularnewline
\hline 
SFU & 1,2 & 3,3 & 4,2{*} & 2,3\tabularnewline
\hline 
UFV & 1,4 & 1,4 & 2,4 & 3,5\tabularnewline
\hline 
UOT & 3,3 & 1,2{*} & 2,3 & 4,2{*}\tabularnewline
\hline 
OUT & -,5 & -,5 & -,5 & -,4\tabularnewline
\hline 
\end{tabular}
\par\end{center}

\begin{center}
\begin{tabular}{|c|c|c|c|c|}
\hline 
 & 1 & 2 & 3 & 4\tabularnewline
\hline 
\hline 
UBC & 1,1{*} & \textcolor{red}{4,1{*}} & \textcolor{red}{2,1{*}} & \textcolor{red}{3,1{*}}\tabularnewline
\hline 
SFU & 1,2 & 3,3 & 4,2{*} & 2,3{*}\tabularnewline
\hline 
UFV & 1,4 & 1,4 & 2,4 & 3,5\tabularnewline
\hline 
UOT & 3,3 & 1,2{*} & 2,3 & \textcolor{red}{4,2{*}}\tabularnewline
\hline 
OUT & -,5 & -,5 & -,5 & -,4\tabularnewline
\hline 
\end{tabular}
\par\end{center}

\begin{center}
\begin{tabular}{|c|c|c|c|c|}
\hline 
 & 1 & 2 & 3 & 4\tabularnewline
\hline 
\hline 
UBC & 1,1{*} & \textcolor{red}{4,1{*}} & \textcolor{red}{2,1{*}} & \textcolor{red}{3,1{*}}\tabularnewline
\hline 
SFU & 1,2 & 3,3 & \textcolor{red}{4,2{*}} & 2,3{*}\tabularnewline
\hline 
UFV & 1,4 & 1,4 & 2,4 & 3,5\tabularnewline
\hline 
UOT & 3,3 & 1,2{*} & 2,3{*} & \textcolor{red}{4,2{*}}\tabularnewline
\hline 
OUT & -,5 & -,5 & -,5 & -,4\tabularnewline
\hline 
\end{tabular}
\par\end{center}

\begin{center}
\begin{tabular}{|c|c|c|c|c|}
\hline 
 & 1 & 2 & 3 & 4\tabularnewline
\hline 
\hline 
UBC & 1,1{*} & \textcolor{red}{4,1{*}} & \textcolor{red}{2,1{*}} & \textcolor{red}{3,1{*}}\tabularnewline
\hline 
SFU & 1,2 & 3,3 & \textcolor{red}{4,2{*}} & 2,3{*}\tabularnewline
\hline 
UFV & 1,4 & 1,4 & 2,4{*} & 3,5\tabularnewline
\hline 
UOT & 3,3 & 1,2{*} & \textcolor{red}{2,3{*}} & \textcolor{red}{4,2{*}}\tabularnewline
\hline 
OUT & -,5 & -,5 & -,5 & -,4\tabularnewline
\hline 
\end{tabular}
\par\end{center}

\textbf{Exercise: }
\begin{enumerate}
\item Find the stable outcome associated with the university proposing deferred
acceptance algorithm for this last example. Are they the same? Can
you see why this means that the stable matching is unique in this
case? Can you see why neither schools or students want to misrepresent
their preferences in this case?
\item Suppose all the students agree on the ranking of the schools (pick
your favorite ranking). What happens. Is there a difference between
the student proposing and the school proposing outcome?
\item As special case of the above occurs when all the schools agree on
the ranking of students, and all the students agree on a ranking of
schools. What happens then? This outcome is called 'assortative matching'.
\item Go back to the problem we discussed above in which students lied about
their preferences when using the school proposing algorithm. Can you
describe one or more Nash equilibrium outcomes having the property
that every students and schools report about their preferences is
a best reply to the reports of the others? How do schools report in
these equilibria?
\end{enumerate}

\subsection*{Indifference}

Lets go back to our original problem in which 3 students rated 3 schools
according to
\[
1\rightarrow UBC\succ UOT\succ SFU
\]
\[
2\rightarrow UBC\succ SFU\succ UOT
\]
\[
3\rightarrow UOT\succ UBC\succ SFU
\]


Now when the students submitted these preferences, lets suppose that
1 and 2 are exactly as stated, but that 2 really doesn't care whether
he goes to UBC or SFU, but strictly prefers both of them to UOT. He
had to fill out something as his favorite choice, so he flipped a
coin and it came out UBC. Furthermore, lets change the preferences
of the schools so that they don't really care who they admit, they
just want to fill their slots. What should you do then?

Recall that the student proposing deferred acceptance algorithm supports
a stable outcome and has the property that students report their preferences
truthfully. To make it work, we needed the schools to reject applications.
We can make them do that by having them run a lottery which ranks
the students. So the administrators in the different schools create
a spreadsheet with the three students listed in one column and then
sort the column randomly. Once they do this they submit these randomized
preferences. They could look like anything, but suppose they were
\[
SFU\rightarrow3\succ1\succ2
\]
\[
UBC\rightarrow2\succ3\succ1
\]
\[
UOT\rightarrow1\succ2\succ3.
\]


Now the program runs, and 2 gets into UBC, 1 gets into UOT and 3 gets
into SFU. This is stable given the stated preferences, but these preferences
are masking a lot of indifference. In fact, if we switched 1 to UBC,
2 to SFU and 3 to UOT, 1 and 3 would be strictly better off and 2
would be no worse than he was initially. Since the schools don't really
care, this new matching would constitute a pareto improvement - not
good.

In this simple case, there is an easy way to resolve this. It provides
an example of something called the 'top cycle' algorithm. In the matching
above $m\left(2\right)=UBC$, $m\left(1\right)=UT$, $m\left(3\right)=SFU$.
The students learn this information. Suppose we now give them the
option to propose one or more schools that they would be willing to
trade for their existing allocation - encouraging them to include
another school if they don't really care whether they are matched
with their existing school or their alternative proposal. The proviso
is that they must be willing to give up their existing school for
the alternative they specify. The system will switch them to the new
school if they can.

Hopefully, student 2 will propose that he is willing to trade UBC
for SFU. Student 1 should propose to trade UBC for UT. Student 3 should
propose trading either UT or UBC for his current allocation at SFU.
The top cycle algorithm takes these proposed trades and goes through
the following routine:
\begin{enumerate}
\item Choose any student, say student 1 and check whether there are any
schools he would like to trade for his current match. Student 1 has
proposed UBC as an alternative to UT. So we start looking for a cycle
using student 1.
\item UBC is owned by student 2, so check whether there is any school that
student 2 is willing to trade for UBC. We find he has proposed SFU
as an alternative he would be willing to trade. So lets add student
2 to our cycle, getting $1\rightarrow2$.
\item SFU is owned by student 3, so check whether there is any school that
3 is willing to trade. One of the schools he has proposed is UBC,
so lets add student 3 to the cycle $1\rightarrow2\rightarrow3$.
\item UBC is owned by student 2 who is willing to trade for UBC, so add
him $1\rightarrow2\rightarrow3\rightarrow2$.
\end{enumerate}
In this last step, we added a student who is already in the list.
This gives us a 'top cycle'. If we continued to apply the steps in
the algorithm above we would cycle back and forth between student
2 and 3. The reason for the cycle is that 2 prefers the school that
3 has and 3 prefers the school that 2 has. So switching them creates
a pareto improvement. We might as well switch them and change the
matching to $m\left(2\right)=SFU$, $m\left(1\right)=UT$, $m\left(3\right)=UBC.$

Notice that we made some students better off, and didn't hurt anyone.
You only propose trades you prefer - you might get them or you might
get nothing. No one is being made any worse off that they were initially.
We might as well ask the question again, are there any schools you
would be willing to trade for your new allocation. All three of them
have schools they prefer - 1 prefers UBC to his new allocation at
UT, 2 also prefers UBC while 3 prefers UT. So lets do it again:
\begin{enumerate}
\item 1 prefers UBC to UT so lets add him to the cycle $1\rightarrow$;
\item UBC is now owned by student 3 who prefers UT, so lets add her to the
cycle $1\rightarrow3$;
\item UT is owned by $1$ who prefers UT so we have $1\rightarrow3\rightarrow1$
- a top cycle.
\end{enumerate}
So student 1 and 3 should switch schools giving $m\left(2\right)=SFU$,
$m\left(1\right)=UBC$, $m\left(3\right)=UT$.

If we try this again we will find the only student we can add to a
cycle is student 2 who prefers UBC, but it is owned by student 1 who
is no longer willing to move - our algorithm comes to an end with
an allocation that is pareto optimal and stable as long as the students
haven't lied about their preferences. In fact, if you recall, we assumed
that student 2 didn't really care whether he got UBC or SFU, so each
of the students actually gets his favorite choice.
\end{document}
